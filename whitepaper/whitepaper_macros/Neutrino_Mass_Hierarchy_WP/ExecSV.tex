The neutrinos remain the most enigmatic of the fundamental fermions and we still don't know the answers to several basic questions:  what is their absolute mass scale?  Do neutrinos violate CP?  Are neutrinos Dirac or Majorana?  
Knowledge of the neutrino mass hierarchy can help to inform each of these questions and is thus a fundamental step towards completion of the Standard Model of particle physics.
%Thus the neutrino mass hierarchy (mass  hierarchy) is one of the fundamental challenges in completing the Standard Model of particle physics.  
Moreover, it may lead to hints of physics beyond the Standard Model, since neutrinos may obtain their mass in a different way than other fundamental fermions.  The neutrino mass hierarchy has implications, as well,  for cosmology and for neutrinoless double beta decay.  
Though these are undeniably fundamental questions, it was outside the scope of this study to evaluate the importance of determining the neutrino mass hierarchy relative to other opportunities on a similar timescale.


Of the experiments considered, only the long baseline technique has demonstrated the ability to measure the mass hierarchy independent of oscillation parameters.  LBNE, in combination with T2K/\NOvA, promises to resolve the mass hierarchy with a significance of more than $3\sigma$ by 2030.   Hyper-Kamiokande can achieve a similar significance on a similar timescale by combining a shorter baseline measurement with atmospheric neutrino data.  The European LAGUNA-LBNO project promises an exceptional sensitivity of greater than ($5\sigma$) on a short timescale ($\approx 1$ year of data) due to its very long baseline, but the project status remains in question.

A variety of other experiments have been proposed with some sensitivity to the mass hierarchy. 
%All of these suffer from important limitations, either in statistics or systematics, or both.  
%Nonetheless, t
These are of interest primarily because some could be completed much more rapidly than long-baseline projects and careful attention to the design of the experiments could give them a reasonable chance of measuring the mass hierarchy.  

%In particular, a couple are limited by difficulties in energy calibration.  
The most viable approaches appear to be reactor neutrinos (JUNO, formerly known as Daya Bay II) and  neutrinos in ice (PINGU at IceCube).  While requiring significant technological advances in detector design and performance, JUNO promises a potential sensitivity of more than  $3\sigma$ ($4\sigma$) assuming current (future $1.5\%$-level) uncertainties on $\Delta m^2_{32}$.  This challenging experiment appears to be on the fast track to approval in China.  PINGU offers excellent statistical sensitivity to the hierarchy, with the primary challenge lying in controlling and evaluating systematic effects.  Sensitivity estimates vary 
and are subject to the choice of oscillation parameters and hierarchy.  In a favorable scenario, a $4\sigma$ measurement could be achieved with 3 years of data; a more conservative analysis finds a $1-5\sigma$ range in sensitivity.
At the time of composing this report, these studies are still being refined.   

Future dark energy experiments such as MS-DESI (formerly BigBOSS),
Euclid, and LSST have the capability to measure the sum of the neutrino masses with precision relevant to the mass hierarchy.  Should the hierarchy be normal and the neutrino masses minimal, MS-DESI could provide an early indication and other dark energy experiments could discern this at a several-sigma level from the power spectrum on a timescale comparable to that for LBNE.

While none of these other experiments, nor current long-baseline oscillation measurements (T2K, \NOvA), is certain to be able to measure the mass hierarchy, one or more of them could do so if oscillation parameters are
 favorable.  With more probability, one might find an indication of the hierarchy at, say, a two-sigma level.   

%The Physics and Nuclear Science Divisions have played central roles in the development of neutrino physics through the KamLAND, SNO, Daya Bay, and IceCube experiments.  We have not previously played a significant role in accelerator-based neutrino experiments.  The success of our previous endeavors was due in part to the enthusiasm of the LBNL scientists who chose to join.  The response of the two divisions to the opportunities laid out in the report should provide the basis for the level of our commitment to the mass hierarchy question. 
