There are a variety of experiments subtly influenced by the neutrino
mass hierarchy.  However, the anticipated sensitivity of these
experiments is unlikely to provide useful information about the
hierarchy.  These include solar neutrino experiments, measurements of
galactic supernovae, neutrinoless double beta decay, and direct
measurement of neutrino masses.   These are summarized in Table~\ref{t:MH1}.

Solar neutrino measurements are sensitive to the sign of $\Delta m^2_{21}$. 
%once the state $\nu_1$ is defined to be the one with the greater $\nu_e$ component.  
While the sensitivity of solar neutrino oscillations
to the sign of the remaining mass splitting, $\Delta m^2_{31}$, is negligible, precision probes of the solar
sector can provide a more detailed understanding of the interaction of neutrinos with matter,
and thus inform terrestrial experiments.

The observation of neutrinos from core-collapse supernovae would be sensitive to the mass
hierarchy via MSW effects in two ways: ``standard'' adiabatic level-crossing effects, and
collective effects in the core. The former is well understood, as evidenced by observations
in the solar neutrino sector, and predicts large rate and systematic shape
distortions, affecting the relative rates observed in different detectors. The collective effects
are less well understood, with large uncertainties and many open questions. Further work
is needed to fully understand these effects, and to determine with confidence that they do
not smear out the standard MSW effects to the point at which a hierarchy determination
becomes impossible. As such, due to the large model uncertainties and the non-predictive
nature of the source, there is an inherent difficulty in using these observations to measure the
hierarchy. Because the rate of supernovae in the galaxy is about one per century, it is also
difficult to anticipate the next experimental opportunity in this field. Instead, knowledge of
the neutrino mass hierarchy would provide valuable input to supernova modeling.

 Current direct neutrino mass measurements do not have sufficient  sensitivity to reach either the
normal or the inverted hierarchy regimes. It is not anticipated that these experiments could
reach mass scales below 0.1 eV in the foreseeable future. The primary purpose for current
experiments is to probe the mass scale in the quasi-degenerate region in a model independent
way. They will also complement neutrinoless double beta decay experiments and cosmological studies down to the $0.2$~eV level.
 
Neutrinoless double beta decay experiments are sensitive to the
neutrino mass hierarchy, but taken alone cannot provide a definitive
measurement unless a clear observation is made in an unambiguous
region of parameter space. The unambiguous region, unfortunately,
occurs at an extremely low value of the decay rate, orders of
magnitude below that accessible to current experiments.  As a
consequence, this is not considered a viable method by which to
determine the neutrino mass hierarchy; instead, the hierarchy would
provide a valuable input to this field, defining the scope for future
experiments. The next-generation experiments, planned for late 2010s
to early 2020s, will aim to have sensitivity sufficient to completely
explore the inverted hierarchy region, independently of nuclear
effects. 

\begin{table}[!htdp]
%\squeezetable
\begin{center}
\caption{Comparison of mass hierarchy (MH) experiments. \label{t:MH1}}
%\begin{tabularx}{\linewidth}{ lllll}
\begin{tabular}{ p{3.5cm }p{3cm }p{3.cm}p{4cm}p{0cm}}
\hline \hline
%{\bf Technique} Experiment  & Timescale   & Scope for mass hierarchy & Other scope & Major concerns &LBL opportunities  \\
{\bf Technique} $\qquad\qquad$ Experiment  & \centering{MH sensitivity} & \centering{Timescale for results}    &  \centering{Major concerns} &  \\
\hline 
\bf Solar \\
%All   & -- & Zero & Neutrino interactions with matter, solar luminosity \& metalicity & No scope for mass hierarchy & Scope for increased involvement (SNO+, future programs) \\
All   & \centering{Zero} & \centering{Ongoing}  &  \centering{No sensitivity to sign of $\Delta m^2_{32}$} &\\
\hline
\bf Supernova \\
% LBNE, large-scale LS, water Cherenkov & Unpredictable & Limited & Astrophysics, neutrino behaviour & Unpredictable timescale, astrophysical uncertainties & Many current involvements \\
 Liquid argon TPC, $\quad\quad\quad$ large-scale LS, $\quad$ water Cherenkov & \centering{Model dependent} & \centering{Unpredictable}  & \centering{ Unpredictable timescale, astrophysical uncertainties} & \\
\hline
\bf Direct mass \\
% All & $\sim$ 2020 & Zero (unless degenerate) & Neutrino mass & Not a mass hierarchy measurement & Already involved in KATRIN  \\ 
 All & \centering{Zero (unless degenerate)} & \centering{$\sim$2020}  &  \centering{Only sensitive in degenerate region }& \\ 
\hline
\multicolumn{3}{l}{\bf Neutrinoless double beta decay }\\
% All & $\sim $2025  & Limited & Lepton-number violation, neutrino nature, neutrino mass & No scope for definitive mass hierarchy measurement  & Already  heavily involved\\
 All & \centering{Limited by Nature}  & \centering{$\sim$2025}   &  \centering{No scope for definitive mass hierarchy measurement}& \\
\hline \hline
\end{tabular}
\end{center}
\end{table}%

