\subsection{Towards Quantitative Understanding - Theory Initiatives}
\label{Sec:Theory}

A large and diverse worldwide theory community is working on the challenges
posed by the discoveries of the experimental heavy ion programs at RHIC and at the LHC.
This community develops theoretical tools suited for the upcoming era of detailed experimental
investigation. It includes, amongst others, lattice QCD groups producing
{\em ab initio} calculations of QCD at  finite temperature and density, high energy physicists and nuclear physicists
working on the embedding of hard partonic processes in a dense nuclear environment, groups advancing
the development of relativistic fluid dynamic simulations of heavy ion collisions and the interfacing of these
simulations with hadronic cascades, field theorists aiming at developing from first principles a description of
the initial conditions of high parton density and their (non-equilibrium) evolution, 
as well as many-body approaches to evaluate spectral and transport properties of QCD matter,
and string theorists contributing
to the exploration of novel strong coupling techniques suited for the description of strongly coupled almost
ideal non-abelian plasmas. This diverse theory community shows all the hallmarks of an active, forward looking
community centered around a mature experimental program with long perspective: it develops improved 
phenomenological tools that address with increased precision and broadened versatility the needs of a
multi-faceted experimental program, and its impact branches to neighboring fields of theoretical physics,
including high energy physics, string theory and astrophysics/cosmology. Here, we highlight only a few 
important recent developments that support these general statements:

\begin{itemize}
\item Development of viscous relativistic fluid dynamic 3+1D simulation tools that pass a broad set of 
theoretical precision tests and that are instrumental in the ongoing program of extracting from the 
experimentally observed flow harmonics and reaction plan correlations material properties of the produced  
quark-gluon plasma. Within the last five years, in a community-wide effort coordinated by, amongst others,
the TECHQM\cite{TECHQM} initiative, these simulation codes were validated against each other. The range of
applicability of these simulations continues to be pushed to further classes of experimental observables.
\item The JET collaboration\cite{JET} has coordinated a similarly broad cross-evaluation of the tools available
for the description of jet quenching in hadron spectra. At the same time, a significant number of 
tools were developed for the simulation of full medium-modified parton showers suited for the modeling
of reconstructed jets. In the coming years, these tools will be the basis for a detailed analysis of jet-medium
interactions.  
\item There is a community-wide effort devoted to extending CGC calculations from LO
 (current phenomenology) to 
NLO~\cite{Balitsky:2008zza,Chirilli:2011km,Stasto:2013cha,Beuf:2014uia,Kang:2014lha}.
Doing NLO calculations in the presence of a non-perturbatively large parton density requires overcoming
qualitatively novel, conceptually challenging issues that are not present in standard {\em in vacuo} NLO calculations in QCD. Within the last
year, the key issues in this program have been addressed by different groups in independent but consistent
approaches, and the field is now rapidly  advancing these calculations of higher precision to a practically
usable level. 
\item In recent years, there have been significant advances in understanding how thermalization occurs in the
initially overoccupied and strongly expanding systems created in heavy ion collisions~\cite{Berges:2013eia,Gelis:2013rba,Kurkela:2014tea}. While some of these
developments are still on a conceptual field theoretical level, there is by now the exciting realization that the thermalization
processes identified in these studies share many commonalities with the problem of dynamically describing the
quenching of jets in dense plasmas~\cite{Blaizot:2013hx,Kurkela:2014tla}. This is likely to open new possibilities for understanding via the detailed measurements
of jet quenching how non-abelian equilibration processes occur in primordial plasmas. 
\item Systematic efforts are being pursued to unravel key properties of QCD matter with heavy-flavor particles. The 
construction of heavy-quark effective theories benefits from increasingly precise information from thermal lattice QCD, to 
evaluate dynamical quantities suitable for phenomenology in heavy-ion collisions (heavy-flavor diffusion coefficient, 
quarkonium spectral properties). This will enable precision tests of low momentum heavy-flavor observables, providing a 
unique window on how in-medium QCD forces vary with temperature.
\end{itemize}

The significant advances listed above document how theory addresses the challenge of keeping at pace with the
experimental development towards more complete and more precise exploration of the hot and dense rapidly 
evolving systems produced in heavy ion collisions. We emphasize that all these research directions show strong 
potential for further theoretical development and improved interfacing with future experimental analyses.
 For instance, the question of how the fluid dynamic evolution can be
interfaced with microscopic probes that are not part of the fluid, such as charm and beauty quarks or jets, and
how the yield of electromagnetic processes can be determined with satisfactory precision within this framework, 
are questions of high phenomenological relevance for the experimental program in the coming years, and
the community is turning now to them. Similar comments apply to the community-wide validation of jet quenching 
event generators, to the necessity of pushing NLO CGC calculations to predictive tools, or to the question of how
characteristic features of thermalization processes can be constrained in an interplay between experiment and
further theoretical development. Continued support of these theory initiatives is needed to optimally exploit the
opportunities arising from the continued experimental analysis of heavy ion collisions and to interface the
obtained insight with the wide-most worldwide physics community. 
