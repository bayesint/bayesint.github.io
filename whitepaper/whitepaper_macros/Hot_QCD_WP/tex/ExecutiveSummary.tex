\section{Executive Summary}
\label{Sec:ExecSummary}

\noindent 
Over the past decade a panoply of measurements made in heavy ion collisions at the Relativistic Heavy Ion Collider (RHIC) and the Large Hadron Collider (LHC), combined with theoretical advances from calculations made in a variety frameworks, have led to a broad and deep knowledge of the properties of hot QCD matter. 
However, this recently established knowledge of what thermal QCD matter does in turn raises 
new questions about how QCD works in this environment. 
High energy nuclear collisions create exploding little droplets of the hottest matter seen anywhere in the universe since it was a few microseconds old. We have increasingly quantitative empirical descriptions of the phenomena manifest in these explosions, and of some key material properties of the matter created in these ``Little Bangs''. In particular, we have determined that the quark-gluon plasma (QGP) created in these collisions is a strongly coupled liquid with the lowest value of specific viscosity ever measured. However, we do not know the precise nature of the initial state from which this liquid forms, and know very little about how the properties of this liquid vary across its phase diagram or how, at a microscopic level, the collective properties of this liquid emerge from the interactions among the individual quarks and gluons that we know must be visible if the liquid is probed with sufficiently high resolution. These findings lead us to the following recommendations:

\noindent {\bf \large Recommendation \#1:}

\noindent
{\bf 
The discoveries of the past decade have posed or sharpened questions that are central to understanding the nature, structure, and origin of the hottest liquid form of matter that the universe has ever seen. As our highest priority we recommend a program to complete the search for the critical point in the QCD phase diagram and to exploit the newly realized potential of exploring the QGP�s  structure at multiple length scales with jets at RHIC and LHC energies. This requires 
\begin{itemize}
\item implementation of new capabilities of the RHIC facility needed to complete its scientific mission: a state-of-the-art jet detector such as sPHENIX and luminosity upgrades for running at low energies,
\item continued strong U.S. participation in the LHC heavy-ion program, and 
\item strong investment  in a broad range of theoretical efforts employing various analytical and computational methods.
\end{itemize}
}

The goals of this program are to 1) measure the temperature and chemical potential dependence of transport properties especially near the phase boundary, 2) explore the phase structure of the nuclear matter phase diagram, 3) probe the microscopic picture of the perfect liquid, and 4) image the high density gluon fields of the incoming nuclei and study their fluctuation spectrum.
These efforts will firmly establish our understanding of thermal QCD matter over a broad range of temperature. However, the precise mechanism by which those temperatures are achieved beginning from the ground state of nuclear matter requires a new initiative dedicated to the study of dense gluon fields in nuclei, which leads to our second recommendation\footnote{This recommendation is made jointly with the Cold QCD Working Group}:

\noindent {\bf \large Recommendation \#2:}

\noindent
{\bf 
A high luminosity, high-energy polarized Electron Ion Collider (EIC) is the U.S. QCD Community's highest priority for future construction
after FRIB.}

The EIC will, for the first time, precisely image the gluons and sea quarks in the proton and nuclei, resolve the proton�s internal structure including the origin of its spin, and explore a new QCD frontier of ultra-dense gluon fields in nuclei at high energy. These advances are made possible by the EIC�s unique capability to collide polarized electrons with polarized protons and light ions at unprecedented luminosity and with heavy nuclei at high energy. The EIC is absolutely essential to maintain U.S. leadership in fundamental nuclear physics research in the coming decades. 



