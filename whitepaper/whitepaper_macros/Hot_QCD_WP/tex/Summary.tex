\section{Summary}
\label{Sec:Summary}
The previous sections of this white paper have outlined the enormous scientific opportunities in the study of thermal QCD that await exploration in the next decade. Realizing these opportunities on a timely basis will require a dedicated program, key elements of which would include

\begin{enumerate}

\item Detector upgrades to enable 
\begin{itemize}
   \item The second phase of the Beam Energy Scan program at RHIC, utilizing the planned increase in low-energy luminosity from electron cooling at RHIC (see Section~\ref{Sec:RHICUpgrades}),
    to explore the phase diagram of nuclear matter,
    measure the temperature and chemical potential dependence of transport properties,
     and continue the search for the QCD critical point. 
   \item State of the art jet measurements at RHIC to understand the underlying degrees of freedom that create the near-perfect liquidity of the QGP.
   \item Extension of the heavy ion capabilities of the LHC detectors in order to study the QGP at the highest temperatures and densities. 
\end{itemize}

\item Commitments to the RHIC campaigns and the LHC heavy ion runs outlined in Section~\ref{Sec:FacilitiesFuture} to ensure timely availability of new data as the enhanced and upgraded detectors become available. 

\item A strong theory effort containing the following items:


\begin{itemize}
\item Strong continued support of the core nuclear theory program supporting university PI's, national lab groups and the national Institute for Nuclear Theory (INT), which in concert generate key ideas that drive the field and train the next generation of students and post-doctoral fellows.

\item Strong continued support of the DOE Early Career Award (ECA) program in Nuclear Theory, as well as the NSF Early Career Development (CAREER) and Presidential Early Career (PECASE) award programs to recognize and promote the careers of the most outstanding young nuclear theorists. 

\item Strong support of expanded computational efforts in nuclear physics, as outlined in the Computational Nuclear Physics white paper. 
%Progress in heavy-ion theory is strongly linked to the availability of a diverse and expanding array of computational resources, including both leadership class and capacity class computational resources. 

\item Continuation and expansion of the Topical Research Collaboration program, since 
%These collaborations are especially valuable where there are several strains of theory developments that need to be coordinated and harnessed to address specific goals. An example of such a successful effort is the JET Topical Collaboration involving the co-ordinated effort of both theorists and experimentalists, as discussed earlier in this document. 
thermal QCD features several outstanding challenges that require the synthesis of a broad range of expertise, and which could strongly benefit from an expansion of the Topical Collaboration program.
\end{itemize}
\end{enumerate}

Pursuing this program over the next decade will consolidate our knowledge of the thermal properties of QCD, the only gauge theory of nature amenable to experimental study in both the strongly and weakly coupled regimes. In addition, it is likely that the new insights that will emerge from this 
process will surprise us, just as the 
initial studies of truly relativistic heavy ion collisions 
led to paradigm shifts in our understanding 
of fundamental aspects of QCD.